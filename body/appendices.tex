\providecommand\onetextpage{}
\renewcommand{\onetextpage}[1]{%
    \clearpage
    \thispagestyle{empty}
    \vspace*{\fill}
    \noindent
    \makebox[\textwidth]{\Huge \textbf{#1} }
    \vfill
    \clearpage
}
\newcounter{nappendix}
\renewcommand{\thenappendix}{\Alph{nappendix}}
\newcommand{\newappendix}[1]{%
    \refstepcounter{nappendix}
    \onetextpage{Appendix \thenappendix}
    \chapter*{Appendix \thenappendix: #1}
    \label{appendix:\thenappendix}
    \addcontentsline{toc}{chapter}{Appendix \thenappendix: #1}
}

%% Edit past this comment line.

\newappendix{Example}
<TIP: Put additional or supplementary information/data/figures in appendices. />

%% Remove the following appendix before publishing your work.

\newappendix{About \LaTeX}
LaTeX (stylized as \LaTeX) is a software system for typesetting documents. LaTeX markup describes the content and layout of the document,
as opposed to the formatted text found in WYSIWYG word processors like Google Docs, LibreOffice Writer, and Microsoft Word.
The writer uses markup tagging conventions to define the general structure of a document,
to stylize text throughout a document (such as bold and italics), and to add citations and cross-references.

LaTeX is widely used in academia for the communication and publication of scientific documents and technical note-taking in many fields,
owing partially to its support for complex mathematical notation. It also has a prominent role in the preparation and publication of books and articles
that contain complex multilingual materials, such as Arabic and Greek.

Overleaf has also provided a 30-minute guide on how you can get started on using \LaTeX. \cite{overleaf:learnlatex}