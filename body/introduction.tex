\chapter{Introduction}
\label{chap:introduction}

\section{Background}
\label{section:background}

In large campuses or complexes with multiple departments or offices within the same area, parking often becomes a common inconvenience for commuters.
Many individuals waste time driving to a parking lot close to their destination, only to find their preferred lot full, forcing them to drive elsewhere.
This not only leads to frustration but also results in unnecessary fuel consumption and traffic congestion.


Studies indicate that cruising for parking can contribute to between 8 and 74 percent of overall traffic congestion and may take anywhere from 3.5 to 14 minutes to secure a spot \cite{shoup:2006}.
This issue is even more frustrating for non-resident visitors unfamiliar with the area, such as those navigating Kasetsart University, where finding available parking can be particularly challenging without prior knowledge of the area layout.
The complexity is further compounded by the road system, which consists of multiple one-way streets, making it difficult for drivers to efficiently navigate between parking areas and forcing them into longer detours when the preferred parking is not available.


Many parking facilities address this challenge by installing parking indicators that display the number of available spots or highlight specific vacant spaces.
These systems help users determine parking availability before entering the lot, reducing unnecessary driving time and congestion.
However, they typically rely on sensors installed in individual parking spots to detect occupancy.
While effective, such sensor-based systems require significant installation and maintenance costs.
As a result, many authorities choose not to implement these systems due to budget constraints, leaving drivers to search for parking without any indicator.

\section{Problem Statement}
\label{section:problem-statement}

A problem statement refers to a clear central issue,
challenge, or question that the project aims to address or explore. It is a
declaration that highlights the specific problem the author intends to examine,
discuss, or solve throughout the course of the project.

\section{Solution Overview}
\label{section:solution-overview}

A software solution overview provides a high-level and
concise description of a software product or system. It serves as an
introduction to the software, offering a glimpse into its key features,
functionalities, and the problems it aims to address. This overview is often
presented in documentation, marketing materials, or other communication
channels to give stakeholders, potential users, or decision-makers a quick
understanding of what the software does and why it is valuable.

\subsection{Features}
\label{subsection:features}

\begin{enumerate}[leftmargin=80pt]
    \item Feature Name: Short Description of Feature
    \item Feature Name: Short Description of Feature
\end{enumerate}

\section{Target User}
\label{section:target-user}

The target user in a software project refers to the specific
group or demographic of individuals for whom the software is designed and
developed. Identifying the target user is a crucial step in the software
development process as it helps the development team tailor the software to
meet the needs, preferences, and requirements of that particular user group.
Understanding the characteristics, behaviors, and expectations of the target
users is essential for creating a user-friendly and effective software solution.

Here are some key aspects related to defining the target user in a software project:

Demographics: This includes factors such as age, gender,
occupation, education level, and other demographic characteristics. Different
age groups or professional backgrounds may have distinct preferences and
requirements when it comes to software usability.

Skill Level: Consideration of the users' technical proficiency and
familiarity with similar software or technology. The level of technical expertise
can influence the complexity of the user interface, the need for tutorials or
documentation, and other user support features.

Industry or Domain: For software solutions designed for specific
industries or domains, understanding the unique challenges, workflows, and
terminology within that industry is crucial. Tailoring the software to meet
industry-specific needs is often necessary.

\section{Benefit}
\label{section:benefit}

Describe potential benefits of your solution.

\section{Terminology}
\label{section:terminology}

Terminology refers to the specific language, jargon, or
specialized vocabulary used to describe concepts, ideas, or subjects within a
particular field or domain. The use of terminology is often essential for clarity
and precision, especially in books that cover technical, scientific, academic, or
specialized topics.