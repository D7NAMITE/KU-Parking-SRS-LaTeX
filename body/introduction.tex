\chapter{Introduction}
\label{chap:introduction}

\section{Background}
\label{section:background}

In large campuses or complexes with multiple departments or offices within the same area, parking often becomes a common inconvenience for commuters.
Many individuals waste time driving to a parking lot close to their destination, only to find their preferred lot full, forcing them to drive elsewhere.
This not only leads to frustration but also results in unnecessary fuel consumption and traffic congestion.


Studies indicate that cruising for parking can contribute to between 8 and 74 percent of overall traffic congestion and may take anywhere from 3.5 to 14 minutes to secure a spot \cite{shoup:2006}.
This issue is even more frustrating for non-resident visitors unfamiliar with the area, such as those navigating Kasetsart University, where finding available parking can be particularly challenging without prior knowledge of the area layout.
The complexity is further compounded by the road system, which consists of multiple one-way streets, making it difficult for drivers to efficiently navigate between parking areas and forcing them into longer detours when the preferred parking is not available.


Many parking facilities address this challenge by installing parking indicators that display the number of available spots or highlight specific vacant spaces.
These systems help users determine parking availability before entering the lot, reducing unnecessary driving time and congestion.
However, they typically rely on sensors installed in individual parking spots to detect occupancy.
While effective, such sensor-based systems require significant installation and maintenance costs.
As a result, many authorities choose not to implement these systems due to budget constraints, leaving drivers to search for parking without any indicator.

\section{Problem Statement}
\label{section:problem-statement}

A problem statement refers to a clear central issue,
challenge, or question that the project aims to address or explore. It is a
declaration that highlights the specific problem the author intends to examine,
discuss, or solve throughout the course of the project.

\section{Solution Overview}
\label{section:solution-overview}

KU-Parking is a mobile application that leverages existing surveillance cameras to detect and display available parking spots in real-time.
The system processes camera feeds through cloud-based computer vision that identifies parking spaces, detects vehicles using YOLO, and determines occupancy through IoU calculations.
Users can view current parking availability and navigate directly to vacant spots, eliminating wasted search time without requiring new infrastructure installation.

% A software solution overview provides a high-level and
% concise description of a software product or system. It serves as an
% introduction to the software, offering a glimpse into its key features,
% functionalities, and the problems it aims to address. This overview is often
% presented in documentation, marketing materials, or other communication
% channels to give stakeholders, potential users, or decision-makers a quick
% understanding of what the software does and why it is valuable.

\subsection{Features}
\label{subsection:features}

\begin{enumerate}[leftmargin=80pt]
    % TODO: to be update just a place holder
    \item Real-time Occupancy Detection: Identifies vacant and occupied parking spots using computer vision and IoU analysis.
    \item Cloud-based Processing: Analyzes camera feeds remotely without requiring on-site hardware installation.
    \item Mobile Navigation: A smartphone application that guides users directly to available parking lots. It integrates with third-party mapping services like Google Maps, providing real-time directions and location information.
    \item Existing Infrastructure Integration: Utilizes already-installed surveillance cameras instead of new sensors.
    \item Parking Availability Dashboard: Displays the number and location of available spots in each parking area.
    \end{enumerate}

\section{Target User}
\label{section:target-user}

Our primary target users are drivers navigating to the Kasetsart University campus, particularly those unfamiliar with the campus parking facilities. Specifically:
\begin{enumerate}[leftmargin=80pt]
\item Students: New or visiting students who are unfamiliar with parking availability patterns across campus and need efficient navigation to classes.
\item Campus Visitors: Guest lecturers, parents, or other visitors who have limited knowledge of campus layout and parking options.
\item Occasional Campus Users: Alumni, part-time students, or attendees of campus events who visit infrequently and need guidance to available parking.
\end{enumerate}
\section{Benefit}
\label{section:benefit}

% Describe potential benefits of your solution.
The KU-Parking solution offers several significant advantages:

\begin{enumerate}[leftmargin=80pt]
    \item Time Savings: Reduces driving time by providing real-time navigation to available parking spots.
    
    \item Cost-Efficiency and Infrastructure Integration: Integrating with existing surveillance camera infrastructure rather than requiring installation of expensive dedicated parking sensors or systems.
    
    \item Improved User Experience: Eliminates the frustration and uncertainty of finding parking, enhancing the overall campus experience for students and visitors.
\end{enumerate}

\section{Terminology}
\label{section:terminology}

Terminology refers to the specific language, jargon, or
specialized vocabulary used to describe concepts, ideas, or subjects within a
particular field or domain. The use of terminology is often essential for clarity
and precision, especially in books that cover technical, scientific, academic, or
specialized topics.
