\chapter{Introduction}
\label{chap:introduction}

\section{Background}
\label{section:background}

A background chapter in a book serves the purpose of providing essential context, information, or history that is relevant
to the overall understanding of the book's subject matter. This chapter typically appears early in the book and aims to set the
stage for the reader by offering background details that are crucial for comprehending the main narrative.

\section{Problem Statement}
\label{section:problem-statement}

A problem statement refers to a clear central issue,
challenge, or question that the project aims to address or explore. It is a
declaration that highlights the specific problem the author intends to examine,
discuss, or solve throughout the course of the project.

\section{Solution Overview}
\label{section:solution-overview}

A software solution overview provides a high-level and
concise description of a software product or system. It serves as an
introduction to the software, offering a glimpse into its key features,
functionalities, and the problems it aims to address. This overview is often
presented in documentation, marketing materials, or other communication
channels to give stakeholders, potential users, or decision-makers a quick
understanding of what the software does and why it is valuable.

\subsection{Features}
\label{subsection:features}

\begin{enumerate}[leftmargin=80pt]
    \item Feature Name: Short Description of Feature
    \item Feature Name: Short Description of Feature
\end{enumerate}

\section{Target User}
\label{section:target-user}

Our primary target users are drivers navigating to the Kasetsart University campus, particularly those unfamiliar with the campus parking facilities. Specifically:
\begin{enumerate}[leftmargin=80pt]
\item Students: New or visiting students who are unfamiliar with parking availability patterns across campus and need efficient navigation to classes.
\item Campus Visitors: Guest lecturers, parents, or other visitors who have limited knowledge of campus layout and parking options.
\item Occasional Campus Users: Alumni, part-time students, or attendees of campus events who visit infrequently and need guidance to available parking.
\end{enumerate}
\section{Benefit}
\label{section:benefit}

% Describe potential benefits of your solution.
The KU-Parking solution offers several significant advantages:

\begin{enumerate}[leftmargin=80pt]
    \item Time Savings: Reduces driving time by providing real-time navigation to available parking spots.
    
    \item Cost-Efficiency and Infrastructure Integration: Integrating with existing surveillance camera infrastructure rather than requiring installation of expensive dedicated parking sensors or systems.
    
    \item Improved User Experience: Eliminates the frustration and uncertainty of finding parking, enhancing the overall campus experience for students and visitors.
\end{enumerate}

\section{Terminology}
\label{section:terminology}

Terminology refers to the specific language, jargon, or
specialized vocabulary used to describe concepts, ideas, or subjects within a
particular field or domain. The use of terminology is often essential for clarity
and precision, especially in books that cover technical, scientific, academic, or
specialized topics.
