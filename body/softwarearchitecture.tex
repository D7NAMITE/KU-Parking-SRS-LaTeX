\chapter{Software Architecture Design}

\section{AI Component}
\label{section:ai-component}

The AI component is a core element of KU Parking. 
It enables parking availability detection by processing streaming video frames from existing surveillance cameras using computer vision techniques. 
The key components include:

\subsection{YOLO-based Vehicle Detection}
YOLO (You Only Look Once) is a real-time object detection algorithm that processes entire images in a single evaluation pass.
YOLO divides images into a grid and simultaneously predicts bounding boxes and class probabilities for detected objects. 
The algorithm's speed and efficiency make it ideal for processing video feeds in near real-time.

\subsection{Parking Space Boundary Detection}
During the initial system setup, edge detection algorithms are used to assist in defining parking space boundaries from camera feeds. 
These algorithms detect significant changes in pixel intensity that typically represent lines or boundaries of parking spaces. 
By applying techniques the system can identify straight lines marking parking spaces, 
which are then verified and finalized by system administrators. Once these boundaries are established during setup, 
they remain fixed for ongoing operations, serving as reference coordinates for the IoU calculations to determine parking space occupancy.


\section{Algorithm}
\label{section:algorithm}
\subsection{Intersection over Union (IoU) for Occupancy Detection}
Intersection over Union (IoU) is a crucial metric in the KU Parking system used to determine whether a parking space is occupied. 
The algorithm works by comparing the overlap between two areas: the predefined parking space boundary and the bounding box of any detected vehicle.

The IoU calculation is mathematically defined as:

\[
\text{IoU} = \frac{\text{Area of Intersection}}{\text{Area of Union}}
\]

Where:
\begin{itemize}
    \item The intersection is the overlapping region between the parking space boundary and vehicle bounding box
    \item The union is the total combined area of both the parking space boundary and vehicle bounding box
\end{itemize}

The IoU value ranges from 0 to 1, where:
\begin{itemize}
    \item IoU = 0 indicates no overlap (parking space is completely empty)
    \item IoU = 1 indicates perfect overlap (parking space perfectly matches the vehicle dimensions)
\end{itemize}

Our algorithm for determining parking space occupancy follows this process:

\begin{enumerate}
    \item For each frame captured by the surveillance camera:
    \begin{itemize}
        \item Apply YOLO to detect vehicles and generate their bounding boxes
        \item For each predefined parking space boundary:
        \begin{itemize}
            \item Calculate the IoU with each detected vehicle bounding box
            \item If any IoU value exceeds a predefined threshold, classify the space as occupied
            \item If no IoU value exceeds the threshold, classify the space as vacant
        \end{itemize}
    \end{itemize}
    
    % \item Apply temporal filtering to reduce false positives:
    % \begin{itemize}
    %     \item Track occupancy state over multiple consecutive frames
    %     \item Only change occupancy status after consistent detection across several frames
    %     \item This prevents flickering status changes due to temporary occlusions or detection errors
    % \end{itemize}
\end{enumerate}

